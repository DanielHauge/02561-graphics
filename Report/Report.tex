\documentclass[a4paper]{article}
\usepackage[utf8]{inputenc}
\usepackage{listings}
\usepackage{parcolumns}
\usepackage{hyperref}
\usepackage{graphicx}
\usepackage{float}
\usepackage{pifont}
\newcommand{\cmark}{\ding{51}}
\newcommand{\xmark}{\ding{55}}
\newcommand{\umark}{\xmark/\cmark}
\newcommand{\dmark}{\cmark/\xmark}
\usepackage{amsmath}
\usepackage{tikz}
\usepackage{makecell}
\usetikzlibrary{shapes.misc, arrows.meta}
\usepackage{array}
\newcolumntype{C}{>$c<$}
\usepackage{color}
\usepackage{fancyhdr}
\usepackage{geometry}
\usepackage[T1]{fontenc}
\usepackage[utf8]{inputenc}
\usepackage{lastpage}
\usepackage{listings}
\usepackage{enumitem}
\usepackage{tabu}
\usepackage[flushleft]{threeparttable}
\usepackage{ulem}
\usepackage{amsmath}
\usepackage{pdfpages}
\usepackage{multicol}
\usepackage{tabu}
\usepackage{parskip}
\usepackage{qtree}
\usepackage{tcolorbox}
\usepackage{scalerel,amssymb}
\def\mcirc{\mathbin{\scalerel*{\circ}{j}}}
\def\msquare{\mathord{\scalerel*{\Box}{gX}}}
\newcommand{\code}[1]{\begin{tcolorbox} \texttt{#1} \end{tcolorbox}}
\def\doubleunderline#1{\underline{\underline{#1}}}
\begin{document}

% COURSE NAME
\title{02561 - Computer Graphics}
\author{
\includegraphics[width=0.15\textwidth]{images/dtu.eps}~\\[1cm]
    DTU - Technical University of Denmark
    \\[0.5cm]
    Date of submission: \today
    \\
}
\date{} % blind date
\color{black}
\maketitle
\begin{center}
{ \huge \bfseries Virtualizing physical object movements}\\

\vspace{.25cm}
Daniel F. Hauge \texttt{(s201186)}\\

% \vspace{.25cm}
% Name \texttt{(s12345678)}\\

\vspace{.25cm}
\end{center}

\begin{abstract}
This project aims to virtualize and display a physical object with it's physical movements on screen. 
The project consists of an implementation which succesfully captures 3 degrees of freedom and uses the movements to display on screen.
\\
{ \begin{center} \bfseries The implementation and lab journal is hosted on: \href{https://grafik.feveile-hauge.dk}{grafik.feveile-hauge.dk}. \end{center}}
\end{abstract}

\medskip
\newpage

\section{Introduction}

- Project introduction, what is it about and more.

\subsection{Problem statement}
- How to capture and display object movements and display it on screen.

\subsection{Motivation \& Usages}
- Virtual / Augmented reality applications, Multi-faceted interface for example 3D modelling work, 3D animation acting (Motion capture)

\section{Method}

\subsection{Capture physical movements}
One of the challenges is to capture the physical movements.
There are different ways of capturing movements, and an often used method is using light.
Light is often used as it is fast, accurate and reliable. Two of the virtual environment head mounted display firms, 
Valve and HTC with their corresponding VRHMD Index and Vive are using light for positioning. 
Their systems require atleast two running base stations, which are essentially casting light for the headset and controllers to be tracked.


Using base stations casting light is extensive and has it's disadvantages, so another way, is to use kinetics.
Most modern smartphones have multiple sensors which can be used for capturing physical movements.

To capture orientation, a gyroscopic sensor can capture angular velocity, 
an accelerometer sensor can provide a reference point for which way is up and down by the earths gravity
and a magnetometer sensor can tell which way is north et cetera. 
All together, they can provide an absolute orientation with respect to the earth. 
Rotating with euler angles have problems such as gimbal lock and weird rotation transitions, quaternions are therefor much prefered over euler angles.
If the previously mentioned sensors are available, the \href{https://nitinjsanket.github.io/tutorials/attitudeest/madgwick#madgwickfilt}{Madgwick Filter algorithm} can be used to determine the orientation with respect to the earth in the form quaternions.
This will capture the 3 rotational motions: Roll, Pitch and Yaw.

To capture displacement, an accelerometer sensor in conjuction with an orientation can provide an absolute displacement.
If sensors are started with a initial velocity $v_0$ of 0, then velocity can be updated with the acceleration $a$ and sensor frequency $t$ in the following equation:
\begin{equation}
    v_1 = v_0 + a \cdot t
\end{equation}

With initial velocity and new velocity, the displacement $d$ can be calculated with the following equation:

\begin{equation}
    d = \frac{v_0 + v_1}{2} \cdot t
\end{equation}

The displacement can then be used to accumulate a total displacement vector which is then used to translate the object in virtual space.
The acceleration sensor in a phone has locked acceleration coordinates, such that:
\begin{itemize}
    \item x running along the width of the device, where right is the positive direction.
    \item y correspond to the height of the device, where up is the positive direction.
    \item z correspond to the depth of the device, where screen face is the positive direction.
\end{itemize}   
Because the coordinates of the acceleration is locked, the displacements calculated from these accelerations will need to be adjusted.
Before accumulating the displacement, they need to be adjusted according to the orientation.
The adjustment can be done by applying/multiplying the quaternion/rotation matrix to the displacement.
Given the rotation matrix $R$ and the displacement $d$, $d$ can be adjusted and a updated total displacement vector $D$ can be calculated.
\begin{equation}
    D_{new} = D_{old} + R \cdot d
\end{equation}
With a initial total displacement of $D_x = D_y = D_z = 0$, the accelerometer can provide a way to capture the displacement movement.
This will capture the 3 transitional motions: Surge, Sway and Heave. 


\subsection{Data transfer}
Because the phone is the sensor, it will be rotated in all different ways and is therefor not used as interface for displaying the virtual environment. 
The phone has the captured sensor data, this data will need to be sent to another device for displaying the movements.
There are multiple wireless options for transfering the sensor data. 
Internet connection through wifi is the choosen method for connecting the device to a more suitable display device.

\subsection{Display}
Displaying rotation can be done at different stages. The rotation could be applied in eye space, such that the object would appear rotated in the correct manner, but everything else would also be rotated the same.
A preferable rotation application would be to world space, ie. the vertecies of the object. 
With the following equation all vertecies $v_i$ can be rotated by an arbitrary amount of quaternions $q_i$:
\begin{equation}
    q_i \cdot ( q_{i-1} \cdot ( \ldots (q_1 \cdot v_i \cdot q_1^{-1}) \ldots ) \cdot q_{i-1}^{-1}  )  \cdot q_i^{-1}
\end{equation}

A convenient way to work with quaternion rotation would be to calculate a rotation matrix.
The rotation matrix can then be used in conjuctions with other transformation matrices. 
The rotational matrix $R$ can be calculated from a quaternion $q$ by the following \href{https://www.euclideanspace.com/maths/geometry/rotations/conversions/quaternionToMatrix/index.htm}{equation}:
\begin{equation}
    R =
    \left[ {\begin{array}{cccc}
        1 - 2\cdot q_y^2 - 2\cdot q_z^2 & 2\cdot q_x\cdot q_y - 2\cdot q_z\cdot q_w & 2\cdot q_x \cdot q_z + 2 \cdot q_y \cdot q_w & 0 \\
        2\cdot q_x \cdot q_y + 2 \cdot q_z \cdot q_w & 1 - 2 \cdot q_x2 - 2\cdot q_z^2 & 2\cdot q_y \cdot q_z - 2 \cdot q_x \cdot q_w & 0 \\
        2\cdot q_x \cdot q_z - 2 \cdot q_y \cdot q_w & 2\cdot q_y \cdot q_z + 2\cdot q_x \cdot q_w & 1 - 2 \cdot q_x^2 - 2 \cdot q_y^2 & 0 \\
        0 & 0 & 0 & 1 \\
    \end{array} } \right]
\end{equation}

Displaying a displacement is achieved by simple translation. 
With a total displacement vector $d$, a translation matrix $T$ can be calculated:
\begin{equation}
    T = \left[ {\begin{array}{cccc}
        1 & 0 & 0 & d_x \\
        0 & 1 & 0 & d_y \\
        0 & 0 & 1 & d_z \\
        0 & 0 & 0 & 1 \\
    \end{array} } \right]
\end{equation}

Transformation with the rotation and translation matrix is the choosen method for moving the object in virtual space.
\section{Implementation}

\subsection{Capturing physical movements}
\subsubsection{Orientation}
To make an implementation that would work for most devices, a widely available API should be used. There is support for \textbf{AbsoluteOrientationSensor} API in 
most modern browsers. This sensor provides an API for capturing orientation of a mobile device. The API outputs quaternion, which is very usefull for the currently choosen method.
To use the phone with the web api for capturing movements, a new page on the website is created for the phone to visit.
When visiting the page from a phone, the sensor data is captured and sent to another device which is visiting the display page.
The sensor is initialized with frequency and reference frame, and a event listener can be added before starting.

The orientation sensor is setup and started with the following code snippet:
\code{orientationSensor = new AbsoluteOrientationSensor(\{ frequency: 60, referenceFrame: 'device' \}); \\
orientationSensor.addEventListener('reading', ev => \{ \\
\hspace*{10mm} const x = orientationSensor.quaternion[0]; \\
\hspace*{10mm} const y = orientationSensor.quaternion[1]; \\
\hspace*{10mm} const z = orientationSensor.quaternion[2]; \\
\hspace*{10mm} const w = orientationSensor.quaternion[3]; \\
\hspace*{10mm} console.log(x, y, z, w); \\
\});\\
orientationSensor.start();
}

On each read, the sensor object has an updated property called \textbf{quaternion}, which is the last captured physical rotational movements.

\subsubsection{Displacement}
Just like orientation, displacement is captured with a web api which is supported in most modern browsers. 
The page that captures phone data is extended to capture acceleration aswell. 
The \textbf{LinearAccelerationSensor} is used for capturing acceleration.
Just like the orientation sensor, it is initialized with a frequency and an event listener can be added.
Each readings contain an acceleration along x, y and z axis.

\code{accelerationSensor = new LinearAccelerationSensor(\{frequency: 60\});\\
accelerationSensor.addEventListener('reading', () => \{\\
\hspace*{10mm} const x = accelerationSensor.x;\\
\hspace*{10mm} const y = accelerationSensor.y;\\
\hspace*{10mm} const z = accelerationSensor.z;\\
\hspace*{10mm} console.log(x, y, z, w);\\
\});\\
accelerationSensor.start();
}

On each read, the sensor object has updated x, y and z properties, which correspond to the accelerations in x, y and z axis.

\subsubsection{Support}
An important implementation detail for using sensor web api's, is that it is required to access the website through a secure connection.
In other words, HTTPS is required and regular HTTP is not adequate.
Another thing is, that if the device does not have adequate sensors, the API's will not be defined in the javascript environment.
Therefor error handling has also been implemented to let a user know if sensor is unavailable and more. 

\subsection{Data transfer}
The data transfer has to happen between 2 clients of the website.
One client is the phone that is visiting the page which setup capture sensors.
Another client is a device that can display the virtual environment.
The implementation is done with websockets. I found a golang and a javascript library from \href{https://socket.io/}{socket.io} that implements websockets.

Using the library, dedicated connection logic is implemented, such that a phone can connect to another client.
There is quite a bit more to how the data transfer implementation is done, but to keep it simple, only the actual transfering of data is covered.
See \textbf{websocket.js} and \textbf{main.go} for full websocket code. 

When a connection is established, the phone can then emit it's readings to the client via websockets.
The following statement is emiting events with orientation sensor readings to the server.

\code{ProjectSockets.Socket.emit("event", \{ id: connectionId, type: "orientation", value: JSON.stringify(orientationSensor.quaternion)\});}

The website server is extended with a websocket server handler:
Listeners for specific channels can then be added. 
The following listener is adding functionality for the server to receive the phone's events and broadcast them to the connected display client.
\code{server.On("event", func(c *gosocketio.Channel, msg Event) string \{ \\
   \hspace*{10mm} if msg.EventType == "connection" \{ \\ 
   \hspace*{20mm}     if observer[msg.SessionId] || !listener[msg.SessionId] \{ \\
   \hspace*{30mm}     return "BAD" \\
   \hspace*{20mm}   \} else \{ \\ 
   \hspace*{30mm}      observer[msg.SessionId] = true \\
   \hspace*{30mm}   connections[c.Id()] = msg.SessionId \\ 
   \hspace*{20mm}    \} \\
   \hspace*{10mm}   \} \\
   \hspace*{10mm} c.BroadcastTo(msg.SessionId, msg.EventType, msg.Value) \\ 
   \hspace*{10mm}    return "OK" \\
\})
}

The following code will setup a listener on the display client that listens for events from the phone through the server. The event contain the readings from the phone.

\code{ProjectSockets.Socket.on("orientation", orientationArray => \{ \\
    \hspace*{10mm}  let quaternion = JSON.parse(orientationArray); \\
    \hspace*{10mm}   if (ProjectSockets.OnOrientationRead != null) \{ \\
        \hspace*{20mm}      ProjectSockets.OnOrientationRead(quaternion); \\
        \hspace*{10mm}   \} \\
        \}); 
}

This listener calls a function \textbf{OnOrientationRead} on every received readings from the phone.
Similar listeners are made with acceleration, connection, disconnection and alignment.
The function can be assigned to let the display device react to new orientations et cetera.


\subsection{Display}
To display the environment, first a virtual environment with a static object is implemented.
Using the same methods and procedures as in worksheet 5 and 10, an object is loaded with a .OBJ file, and a quad is loaded with a texture for looks.

To rotate the object based on the phones readings, a variable holding a quaternion is declared and updated by each readings from the phone.
The following code assigns the \textbf{OnOrientationRead} function to set a reference orientation and update the quaternion variable with respect to the reference.
\code{ProjectSockets.OnOrientationRead = ori => \{ \\
\hspace*{10mm} const newOri = [ori[0], ori[1], ori[2], ori[3]] \\
\hspace*{10mm} if (Project.refSet == undefined) \{ \\
\hspace*{20mm} Project.refSet = true; \\
\hspace*{20mm} Project.q\_rot\_ref.set(newOri); \\
\hspace*{10mm} \} \\
\hspace*{10mm} Project.q\_rot.set(newOri); \\
\hspace*{10mm} Project.q\_rot.multiply(Project.q\_rot\_ref); \\
\}
}

The quaternion.js contain a function for calculating the matrix, this is used when drawing the phone:

\code{const rot = new Matrix4(); \\
const mat4 = Project.q\_rot.get\_mat4(); \\
const flattened = flatten(mat4); \\
rot.set({elements: flattened}); \\
Project.g\_modelMatrix = Project.g\_modelMatrix.multiply(rot);
}

The phone readings are now able to rotate a model matrix which is then used for drawing an object in a virtual environment.
There are more implementation details which involves perspective alignment with the screen and more, these will not be covered to keep it succinct.
See \textbf{project.js} for full code on display/drawing.

To draw the displacement, an attempt with similar approach to orientation is done.
The following code is assigning the \textbf{OnAccelerationRead} to calculate a new total displacement vector.

\code{const Displacement = \{x:0, y:0, z:0\}; \\
const Velocity = \{x:0, y:0, z:0\}; \\
const DeltaT = 1/15; \\
ProjectSockets.OnAccelerationRead = Acceleration => \{ \\
\hspace*{10mm}   const vX = Velocity.x + Acceleration.x * DeltaT \\
\hspace*{10mm}  const vY = Velocity.y + Acceleration.y * DeltaT \\
\hspace*{10mm}  const vZ = Velocity.z + Acceleration.z * DeltaT \\
\hspace*{10mm}   Displacement.x = 0.5 * (vX + Velocity.x) * DeltaT \\
\hspace*{10mm}  Displacement.y = 0.5 * (vY + Velocity.y) * DeltaT \\
\hspace*{10mm}   Displacement.z = 0.5 * (vZ + Velocity.z) * DeltaT \\
\hspace*{10mm}   Velocity.x = vX; \\
\hspace*{10mm}   Velocity.y = vY; \\
\hspace*{10mm}   Velocity.z = vZ; \\
\hspace*{10mm}   Project.PD.x =  Displacement.x; \\
\hspace*{10mm}   Project.PD.y =  Displacement.y; \\
\hspace*{10mm}   Project.PD.z =  Displacement.z;\\
    \} \\
}

A translation matrix is then multiplied to the model matrix:

\code{Project.g\_modelMatrix.translate(\\ \hspace*{10mm} Project.Phone\_X +  Project.PD.x, \\ \hspace*{10mm} Project.Phone\_Y + Project.PD.y, \\ \hspace*{10mm} Project.Phone\_Z + Project.PD.z);}

Unfortunately, the implementation for displaying displacement is not working as intended.
It seems as through acceleration is not accurate, as moving the phone from one stationary position to another, the velocity is not zero when the phone is done moving.
The acceleration or deacceleration reads must have been larger to produce a non zero velocity.
A different method could perhaps be better at implementing the solution.
\section{Solution}
- The solution is this and that.
With video.

\subsection{Trying the solution}
- Instructions.
\section{Conclusion \& Discussion}

Physical movements can be captured with a modern smartphone's sensors.
An accelerometor, magnetometer and gyroscopic sensor can be used to measure kinetics of the rigid body.
Measurements from the sensors can be transfered to a different device via a internet connection with the help from websockets.
The data is used to calculate transformations matrices, which are then used during rendering to display the movement in virtual space.

The solution succesfully captures 3 degrees of freedom with the rotational motions, and displays the movement on a html canvas.
Unfortunately the remaining 3 degrees of freedom is not succesfully captured and displayed.
The method choosen for translational motions is sensitive to accuracy, and perhaps a different method would have been better.

Regardless of missing degrees of freedom, I still think the project is cool and interesting to play with.


% Credits: https://open3dmodel.com/3d-models/samsung-galaxy-s20_477030.html

\end{document}







