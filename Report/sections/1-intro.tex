\section{Introduction}
With ever growing computing power at disposal, more opportunities for cool technology emerges. One such technology that has just recently picked up speed, 
is virtual reality and augmented reality. I postulate that virtual reality is by no means a matured complete technology, as only two human senses has convincingly been implemented.
The cutting edge VR technology still mostly require all movements to happen in actual reality, ie. movements happen in reality and are captured, virtualized and rendered in VR.
This project will try to tackle the problem of capturing movements to display physical movements in a virtual environment.

\subsection{Problem statement}
How can physical movement be captured and displayed in a virtual environment like on a HTML canvas.

\subsection{Motivation \& Usages}
As previusly mentioned, VR technology still require bodily movements to happen in reality.
So capturing physical movements is currently still relevant within VR technology.
Just like VR, augmented reality also benefit from being able to capture physical movements to be used for displaying graphics.

Capturing physical movements is also relevant in film making.
Movies like Lord of the Rings and Avatar, have used motion capture technology to great success. 

Apart from being used for rendering graphics,
capture of movements can be used in new computer interfaces, 
like some sort of gesture controller.  