\section{Conclusion \& Discussion}

Physical movements can be captured with a modern smartphone's sensors.
An accelerometor, magnetometer and gyroscopic sensor can be used to measure kinetics of the rigid body.
Measurements from the sensors can be transfered to a different device via a internet connection with the help from websockets.
The data is used to calculate transformations matrices, which are then used during rendering to display the movement in virtual space.

The solution succesfully captures 3 degrees of freedom with the rotational motions, and displays the movement on a html canvas.
Unfortunately the remaining 3 degrees of freedom is not succesfully captured and displayed.
The method choosen for translational motions is sensitive to accuracy, and perhaps a different method would have been better.

Regardless of missing degrees of freedom, I still think the project is cool and interesting to play with.