\newpage
\section{Results}
The solution is a website that when accessed with a phone and another display device, can capture rotational motion of the phone and display it's movements on a html canvas.
Unfortunately, translational motion is not succesfully captured and displayed.

The following link is a youtube video of a demo of the solution in action: \href{https://youtu.be/8K5ufYZokSE}{https://youtu.be/8K5ufYZokSE}  

In order to try the solution, a phone with a accelerometor sensor, gyroscopic sensor and magnetometer sensor is required.
Also another device for displaying and phone with adequate browser that support these sensors.
To try the solution, follow these steps:
\begin{itemize}
    \item With display device (ex. computer) navigate with a browser to the running instance of the solution or \href{https://grafik.feveile-hauge.dk}{https://grafik.feveile-hauge.dk} (https is important)
    \item With phone, navigate to \textbf{URL/c?id=XXX} where XXX is substituted with code on the additional device.
    \item On phone, click the "Start sensor" button.
    \item Put phone on stable surface pointing towards the display device and click "Realign" button. 
\end{itemize}

Following these steps will hopefully demonstrate how the solution is able to capture physical movements to be displayed virtually.

\textit{To run an instance of the solution locally. Have golang installed and run "go run main.go" in a terminal at root directory of the extracted zip, with a system environment variable "development" set to true. }
